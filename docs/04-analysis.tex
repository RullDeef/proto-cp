\chapter{Аналитический раздел}

\section{Актуальность}

Одной из наиболее актуальных задач в современном мире является обеспечение эффективного взаимодействия между удаленными пользователями. Бессерверные аудио-видео конференции представляют собой перспективное решение, позволяющее организовывать виртуальные встречи без необходимости использования централизованных серверов.

Традиционные подходы к организации конференц-связи, основанные на клиент-серверной архитектуре, сталкиваются с рядом ограничений. Во-первых, они требуют наличия мощной серверной инфраструктуры, которая должна обрабатывать весь трафик участников, что приводит к высоким эксплуатационным расходам. Во-вторых, централизованные системы являются уязвимыми к сбоям и атакам, так как отказ сервера может привести к полной потере связи между пользователями. В-третьих, такие решения зачастую ограничены в масштабируемости, что затрудняет их применение в крупномасштабных распределенных средах.

Бессерверные аудио-видео конференции, в свою очередь, обладают рядом преимуществ. Они позволяют организовывать виртуальные встречи напрямую между участниками, без необходимости использования централизованных серверов. Это повышает отказоустойчивость системы, поскольку отказ одного из клиентов не приводит к полной потере связи. Кроме того, данный подход способствует снижению эксплуатационных расходов и упрощает процесс масштабирования.

Разработка протокола для организации бессерверных аудио-видео конференций является важной научно-технической задачей, имеющей высокую практическую значимость. Результаты исследования могут найти применение в различных областях, таких как дистанционное обучение, удаленная работа, проведение виртуальных совещаний и конференций, а также во многих других сценариях, требующих эффективного взаимодействия между удаленными участниками.

\section{Потоковая передача данных в реальном времени}

Говоря о аудио-видео конференциях, следует выделить две основные составляющие потоков данных -- аудио поток и видео поток. Аудио поток, как правило, имеет более высокий приоритет по сравнению с видео потоком, поскольку бесперебойная передача звука является критически важной для обеспечения качественного взаимодействия участников.

Основными протоколами транспортного уровня, используемыми при передаче данных в режиме реального времени, являются TCP, UDP. На прикладном уровне часто используются такие протоколы как RTP (Real-Time Transport Protocol  \cite{rtp}), RTCP (Real-Time Transport Control Protocol  \cite{rtp}), SIP (Session Initiation Protocol \cite{sip}), H.323 \cite{h323}.

\section{Распределение вычислений}

При централизованном проведении конференций основная нагрузка ложится на сервер, в то время как при проведении бессерверных конференций она распределяется по клиентам. Для обеспечения наилучшего QoS необходимо учитывать вычислительные возможности каждого клиента, а также пропускную способность каналов, объединяющих их.

В бессерверной архитектуре каждый участник конференции выступает в качестве автономного узла, ответственного за обработку и передачу мультимедийного контента. Это требует от клиентских устройств достаточной вычислительной мощности для кодирования, декодирования и синхронизации аудио, видео и данных в режиме реального времени.

\begin{equation}
  \textit{Perf}_{client} = f(CPU, GPU, RAM)
\end{equation}

Пропускная способность каналов связи также играет критическую роль, определяя максимально возможное качество и разрешение передаваемых потоков. Недостаточная пропускная способность может привести к снижению качества и стабильности соединения, а также к возникновению задержек и потерь пакетов.

\begin{equation}
  \textit{Bandwidth}_{channel}=f(\textit{network~speed},\textit{distance},\textit{noises})
\end{equation}

Для достижения оптимального QoS в бессерверных конференциях необходимо динамически адаптировать параметры кодирования и передачи мультимедийных потоков к возможностям каждого участника. Это может быть реализовано с помощью механизмов согласования характеристик сессии и адаптивного управления скоростью передачи данных.

\section{Микширование}

При большом числе участников в бессерверных видеоконференциях имеет смысл применять микширование передаваемых видео потоков. Это позволяет уменьшить общий объем передаваемых данных, снизив нагрузку на каналы связи, но при этом требует значительных вычислительных ресурсов от клиентских устройств.

Процесс микширования видео можно представить в виде следующих основных шагов:

\begin{enumerate}
\item Получение видео кадров от соседних узлов.
\item Декодирование полученных кадров.
\item Композиция единого кадра путем пространственного объединения (склеивания) исходных кадров.
\item Кодирование скомпонованного кадра.
\item Передача закодированного кадра всем соседним узлам.
\end{enumerate}

При этом, на этапе формирования общего кадра необходимо учитывать, нужен ли тот или иной видео поток каждому из соседних участников. Это позволяет оптимизировать состав композитного кадра и снизить вычислительную нагрузку.

Микширование аудио потоков может быть реализовано схожим образом, с формированием раздельных аудио каналов для каждого участника конференции. Это требует дополнительных вычислений, связанных с микшированием, кодированием и передачей аудио данных.

Эффективность микширования определяется соотношением выигрыша в объеме передаваемых данных и затратами вычислительных ресурсов на клиентских устройствах. Для поддержания высокого качества и низкой задержки передачи мультимедийного контента необходим тщательный подбор алгоритмов микширования и кодирования с учетом возможностей участников конференции.

\section{Устойчивость конференции}

Отключение ключевого участника конференции, через которого проходит большое число потоков данных от других участников, является одной из проблем распределенных вычислений. Решением этой проблемы может послужить механизм, обеспечивающий поддержание <<запасных>> каналов передачи между клиентами.

\section{Защита информации}

Для защиты конференции необходимо использовать защищенные каналы передачи. На каждое соединение инициализируется собственная сессия, вместо того, чтобы использовать одну сессию на всех участников конференции.

Плюсы такого подхода:
\begin{itemize}[label=---]
  \item Повышение безопасности за счет использования индивидуальных криптографических ключей для каждого соединения.
  \item Улучшение масштабируемости системы, так как каждое соединение обрабатывается независимо.
  \item Более гибкое управление доступом и привилегиями участников конференции.
\end{itemize}

Минусы такого подхода:
\begin{itemize}[label=---]
  \item Увеличение вычислительной нагрузки на клиентские устройства, связанное с необходимостью поддержания множества криптографических сессий.
  \item Более высокие требования к пропускной способности канала связи, обусловленные необходимостью передачи дополнительных данных, связанных с криптографической защитой.
\end{itemize}

В программной реализации протокола применяется библиотека OpenSSL \cite{openssl} для осуществления защищенных соединений. Использование OpenSSL позволяет обеспечить совместимость с широким спектром криптографических алгоритмов, а также реализовать такие механизмы безопасности, как аутентификация, шифрование и целостность передаваемых данных. Кроме того, OpenSSL предоставляет средства для управления криптографическими ключами, что упрощает интеграцию защищенных соединений в общую архитектуру системы.
